\documentclass{proposal}
\usepackage{graphicx}
\usepackage[square,numbers]{natbib}
\usepackage{listing}
\usepackage{chngcntr}
\usepackage[hyphens]{url}
\usepackage{hyperref}

\begin{document}

    \counterwithin{listing}{section}


%%%%%%%%%%%%%%%%%%%%%%%%%%%%%%%%%%%%%%%%%%%%%%%%%%%%%%%%%%%%%%%%%%%%%%%%%%%%%%%%%%%%%%%%%%%%%%%%%%%%%%%%%%%%%%%%%%%%%%%%


    \title{ptr-tidy: Automatic refactoring of raw pointers in C++}
    \author{Artem Usov}
    \date{18 December 2020}
    \maketitle
    \tableofcontents
    \newpage


%%%%%%%%%%%%%%%%%%%%%%%%%%%%%%%%%%%%%%%%%%%%%%%%%%%%%%%%%%%%%%%%%%%%%%%%%%%%%%%%%%%%%%%%%%%%%%%%%%%%%%%%%%%%%%%%%%%%%%%%


    \section{Introduction}\label{sec:intro}

    \subsection{Difficulties with Memory Management}\label{subsec:difficulties-with-memory-management}

    % TODO insert info about what memory management/errors are

    Correct memory management is a difficult problem.

    \emph{How difficult one might ask?}

    In a presentation by Matt Miller, a security engineer at Microsoft~\cite{Miller2019}, it is shown that around 70\% of vulnerabilities that are addressed through a security updates are due to memory safety issues.
    This is at an industry-leading company who are renowned for hiring top talent.
    Such memory safety vulnerabilities can be exploited, and due to new regulations such as the GDPR, these issues are more commonly exposed to the general public and punished by regulators.
    Such an example is a fine of £20m fine for a British Airways data breach by the British Information Commissioner's Office~\cite{ICO2020}.

    \subsection{Possible Solutions to Memory Management}\label{subsec:possible-solutions-to-memory-management}

    Memory safety errors occur when languages are used that place the tasks of memory management with the programmer, such as C and C++.
    However simply not using these languages is not an option since their inherent danger makes them highly performant, and therefore the only option for a project such as a web browser.

    \emph{What can be done then to address this issue?}

    % TODO add Rust performance benchmark reference

    Engineers at both Microsoft~\cite{Thomas2019} and Mozilla~\cite{Hostfelt2019} converge on Rust~\cite{Balasubramanian2017} as a possible solution.
    Rust is a systems language that offers the same performance as C and C++, however its linear type system and memory ownership model also guarantee memory safety.
    The rewriting of a program in a new language is a colossal undertaking, especially in language that has a reputation for being difficult to learn.
    We therefore propose an alternative partial solution to this problem.


%%%%%%%%%%%%%%%%%%%%%%%%%%%%%%%%%%%%%%%%%%%%%%%%%%%%%%%%%%%%%%%%%%%%%%%%%%%%%%%%%%%%%%%%%%%%%%%%%%%%%%%%%%%%%%%%%%%%%%%%


    \section{Statement of Problem}\label{sec:statement-of-problem}

    \subsection{Using Smart Pointers for Memory Management}\label{subsec:using-smart-pointers-for-memory-management}

    Generally pointers are used to give the program access to a resource that cannot be directly included in the program itself, such as a file or a thread.
    However Stroustrup claims that pointers to objects allocated on the free store is dangerous and a 'plain old pointer', or raw pointer as we will refer to them, should not be used to represent ownership~\cite{Stroustrup2018}.

    \begin{listing}
        \begin{verbatim}
            void foo(int x) {
                Shape *p = new Circle{Point{0,0},10};
                // ...
                if (x<0) throw Bad_x{}; // potential leak
                if (x==0) return; // potential leak
                // ...
                delete p;
            }
        \end{verbatim}
        \caption{Example of memory leaks using manual management.}
        \label{fig:manual-leak}
    \end{listing}

    In \autoref{fig:manual-leak} we can see that when using a raw pointer, there are two cases in which the programmer will never free the allocated memory.
    Instead we can \emph{smart pointers}~\cite{Dimov2003} and the concept of \emph{RAII (Resource Acquisition Is Initialisation)} to create resource handles which automatically eliminate resource leaks with no added overhead.
    The equivalent code using \emph{unique pointers} can be seen in \autoref{fig:first-example-unique}.
    In the comments, we can see all the locations where \emph{p} exits the scope of the function and where the compiler will insert delete statements, now ensuring that the resource will never get leaked.

    \begin{listing}
        \begin{verbatim}
            void foo(int x) {
                auto p = std::make_unique<Circle>(Point{0,0},10);
                // ...
                if (x<0) throw Bad_x{}; // delete p inserted here
                if (x==0) return; // delete p inserted here
                // ...
                // delete p inserted here
            }
        \end{verbatim}
        \caption{Example of using a unique pointer to manage memory.}
        \label{fig:first-example-unique}
    \end{listing}

    There also exists a \emph{shared pointer} for resources which may not have a single unique owner.
    Reference counting is used to ensure that after the last owner loses access to the resource, it will finally be freed.
    Shared pointers are suitable to be used in most situations, except in structures which have cyclic references \autoref{}.

    % TODO add reference of cycle analysis further in paper

    \subsection{Do Smart Pointer Solve Memory Management?}\label{subsec:do-smart-pointer-solve-memory-management?}


%%%%%%%%%%%%%%%%%%%%%%%%%%%%%%%%%%%%%%%%%%%%%%%%%%%%%%%%%%%%%%%%%%%%%%%%%%%%%%%%%%%%%%%%%%%%%%%%%%%%%%%%%%%%%%%%%%%%%%%%


    \section{Background Survey}\label{sec:background-survey}

    Present an overview of relevant previous work including articles, books, and existing software products.
    Critically evaluate the strengths and weaknesses of the previous work.


%%%%%%%%%%%%%%%%%%%%%%%%%%%%%%%%%%%%%%%%%%%%%%%%%%%%%%%%%%%%%%%%%%%%%%%%%%%%%%%%%%%%%%%%%%%%%%%%%%%%%%%%%%%%%%%%%%%%%%%%


    \section{Proposed Approach}\label{sec:proposed-approach}

    State how you propose to solve the software development problem.
    Show that your proposed approach is feasible, but identify any risks.


%%%%%%%%%%%%%%%%%%%%%%%%%%%%%%%%%%%%%%%%%%%%%%%%%%%%%%%%%%%%%%%%%%%%%%%%%%%%%%%%%%%%%%%%%%%%%%%%%%%%%%%%%%%%%%%%%%%%%%%%


    \section{Work Plan}\label{sec:work-plan}

    Show how you plan to organize your work, identifying intermediate deliverables and dates.


%%%%%%%%%%%%%%%%%%%%%%%%%%%%%%%%%%%%%%%%%%%%%%%%%%%%%%%%%%%%%%%%%%%%%%%%%%%%%%%%%%%%%%%%%%%%%%%%%%%%%%%%%%%%%%%%%%%%%%%%


    \bibliographystyle{plainnat}
    \bibliography{proposal}
\end{document}
